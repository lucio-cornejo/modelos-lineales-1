% Options for packages loaded elsewhere
\PassOptionsToPackage{unicode}{hyperref}
\PassOptionsToPackage{hyphens}{url}
\PassOptionsToPackage{dvipsnames,svgnames,x11names}{xcolor}
%
\documentclass[
  letterpaper,
  DIV=11,
  numbers=noendperiod]{scrartcl}

\usepackage{amsmath,amssymb}
\usepackage{iftex}
\ifPDFTeX
  \usepackage[T1]{fontenc}
  \usepackage[utf8]{inputenc}
  \usepackage{textcomp} % provide euro and other symbols
\else % if luatex or xetex
  \usepackage{unicode-math}
  \defaultfontfeatures{Scale=MatchLowercase}
  \defaultfontfeatures[\rmfamily]{Ligatures=TeX,Scale=1}
\fi
\usepackage{lmodern}
\ifPDFTeX\else  
    % xetex/luatex font selection
\fi
% Use upquote if available, for straight quotes in verbatim environments
\IfFileExists{upquote.sty}{\usepackage{upquote}}{}
\IfFileExists{microtype.sty}{% use microtype if available
  \usepackage[]{microtype}
  \UseMicrotypeSet[protrusion]{basicmath} % disable protrusion for tt fonts
}{}
\makeatletter
\@ifundefined{KOMAClassName}{% if non-KOMA class
  \IfFileExists{parskip.sty}{%
    \usepackage{parskip}
  }{% else
    \setlength{\parindent}{0pt}
    \setlength{\parskip}{6pt plus 2pt minus 1pt}}
}{% if KOMA class
  \KOMAoptions{parskip=half}}
\makeatother
\usepackage{xcolor}
\setlength{\emergencystretch}{3em} % prevent overfull lines
\setcounter{secnumdepth}{5}
% Make \paragraph and \subparagraph free-standing
\makeatletter
\ifx\paragraph\undefined\else
  \let\oldparagraph\paragraph
  \renewcommand{\paragraph}{
    \@ifstar
      \xxxParagraphStar
      \xxxParagraphNoStar
  }
  \newcommand{\xxxParagraphStar}[1]{\oldparagraph*{#1}\mbox{}}
  \newcommand{\xxxParagraphNoStar}[1]{\oldparagraph{#1}\mbox{}}
\fi
\ifx\subparagraph\undefined\else
  \let\oldsubparagraph\subparagraph
  \renewcommand{\subparagraph}{
    \@ifstar
      \xxxSubParagraphStar
      \xxxSubParagraphNoStar
  }
  \newcommand{\xxxSubParagraphStar}[1]{\oldsubparagraph*{#1}\mbox{}}
  \newcommand{\xxxSubParagraphNoStar}[1]{\oldsubparagraph{#1}\mbox{}}
\fi
\makeatother


\providecommand{\tightlist}{%
  \setlength{\itemsep}{0pt}\setlength{\parskip}{0pt}}\usepackage{longtable,booktabs,array}
\usepackage{calc} % for calculating minipage widths
% Correct order of tables after \paragraph or \subparagraph
\usepackage{etoolbox}
\makeatletter
\patchcmd\longtable{\par}{\if@noskipsec\mbox{}\fi\par}{}{}
\makeatother
% Allow footnotes in longtable head/foot
\IfFileExists{footnotehyper.sty}{\usepackage{footnotehyper}}{\usepackage{footnote}}
\makesavenoteenv{longtable}
\usepackage{graphicx}
\makeatletter
\newsavebox\pandoc@box
\newcommand*\pandocbounded[1]{% scales image to fit in text height/width
  \sbox\pandoc@box{#1}%
  \Gscale@div\@tempa{\textheight}{\dimexpr\ht\pandoc@box+\dp\pandoc@box\relax}%
  \Gscale@div\@tempb{\linewidth}{\wd\pandoc@box}%
  \ifdim\@tempb\p@<\@tempa\p@\let\@tempa\@tempb\fi% select the smaller of both
  \ifdim\@tempa\p@<\p@\scalebox{\@tempa}{\usebox\pandoc@box}%
  \else\usebox{\pandoc@box}%
  \fi%
}
% Set default figure placement to htbp
\def\fps@figure{htbp}
\makeatother

\KOMAoption{captions}{tableheading}
\makeatletter
\@ifpackageloaded{caption}{}{\usepackage{caption}}
\AtBeginDocument{%
\ifdefined\contentsname
  \renewcommand*\contentsname{Table of contents}
\else
  \newcommand\contentsname{Table of contents}
\fi
\ifdefined\listfigurename
  \renewcommand*\listfigurename{List of Figures}
\else
  \newcommand\listfigurename{List of Figures}
\fi
\ifdefined\listtablename
  \renewcommand*\listtablename{List of Tables}
\else
  \newcommand\listtablename{List of Tables}
\fi
\ifdefined\figurename
  \renewcommand*\figurename{Figure}
\else
  \newcommand\figurename{Figure}
\fi
\ifdefined\tablename
  \renewcommand*\tablename{Table}
\else
  \newcommand\tablename{Table}
\fi
}
\@ifpackageloaded{float}{}{\usepackage{float}}
\floatstyle{ruled}
\@ifundefined{c@chapter}{\newfloat{codelisting}{h}{lop}}{\newfloat{codelisting}{h}{lop}[chapter]}
\floatname{codelisting}{Listing}
\newcommand*\listoflistings{\listof{codelisting}{List of Listings}}
\makeatother
\makeatletter
\makeatother
\makeatletter
\@ifpackageloaded{caption}{}{\usepackage{caption}}
\@ifpackageloaded{subcaption}{}{\usepackage{subcaption}}
\makeatother

\usepackage{bookmark}

\IfFileExists{xurl.sty}{\usepackage{xurl}}{} % add URL line breaks if available
\urlstyle{same} % disable monospaced font for URLs
\hypersetup{
  pdftitle={Trabajo final},
  pdfauthor={Lucio Enrique Cornejo Ramírez},
  colorlinks=true,
  linkcolor={blue},
  filecolor={Maroon},
  citecolor={Blue},
  urlcolor={Blue},
  pdfcreator={LaTeX via pandoc}}


\title{Trabajo final}
\author{Lucio Enrique Cornejo Ramírez}
\date{2025-06-16}

\begin{document}
\maketitle

\renewcommand*\contentsname{Table of contents}
{
\hypersetup{linkcolor=}
\setcounter{tocdepth}{3}
\tableofcontents
}

\section{Introducción}\label{introducciuxf3n}

\subsection{Marco del problema}\label{marco-del-problema}

Entre los costos que más desastibilizan económicamente a las personas,
se encuentra el pago por procedimientos médicos. Estos precios pueden
variar en gran medida dependiendo de características del paciente, como
detallaremos más adelante.

En ese sentido, resulta de gran valor predecir adecuadamente el costo
que un seguro médico cubrirá, respecto a un procedimiento médico. Para
un paciente, aquella predicción puede servir para que planifique qué
tanto sería desestabilizado económicamente debido a algún tipo de
procedimiento particular. Por otro lado, también para las aseguradoras
resulta útil aquellas predicciones, ya que pueden anticipar qué tanto
dinero estarían perdiendo por el monto a cubrir de la operación; además,
con ese conocimiento pueden monitorear mejor qué pedidos de cobertura
resultan anómalos, potencialmente fraudulentos.

\subsection{Plan de modelamiento}\label{plan-de-modelamiento}

Anteriormente, hemos planteado como variable por predecir a la cantidad
monetaria \(S\) que la aseguradora de un paciente cubrirá debido a un
procedimiento médico. Note que, fijando el precio \(P\) de un
procedimiento médico, tal predicción equivale a predecir la cantidad
\(P - S\) que llegaría a pagar el paciente, habiendo descontado lo que
cubre su aseguradora.

En ese sentido, el \textbf{carácter por modelar} es aquella variable
aleatoria \(P - S\), monto que paga un paciente por un tratamiento
médico, tras aplicarse el descuento de su aseguradora.

Para el modelamiento, se considerará los gastos del hospital debido al
procedimiento médico, además de las siguientes características del
paciente:

\begin{itemize}
\tightlist
\item
  Sexo.
\item
  Si fuma o no.
\item
  Región de la que provee.
\item
  Edad.
\item
  Índice de masa corporal.
\item
  Cantidad de hijos e hijas.
\item
  Costo médico que pagaría en caso no se aplicase seguro médico.
\item
  Número de procedimientos pasados.
\item
  Número de pasos que realizó en cierto día.
\item
  Número de veces que ha sido hospitalizado.
\item
  Salario anual.
\end{itemize}

\subsection{Datos/Observaciones}\label{datosobservaciones}

Las observaciones que consideraremos para este proyecto fueron
descargadas de este sitio
\href{https://www.kaggle.com/datasets/shubhamsingh57/ml-model-practice-linear-regression/data}{web}.

Estos datos son de distintos pacientes que recibieron algún tipo de
tratamiento médico, de los cuales se tienen variables recopiladas, como
edad, sexo, si fuma o no, etc. Así, descartamos que los datos consistan
de una serie de tiempo.

No obstante, aquella página web no provee información más específica
sobre el origen de los datos. Por ejemplo, si han sido recopilados en un
único hospital, o en diversos hospitales, pero de qué país, etc.

Aún así, en esta investigación, no solo se considera la predicción de la
variable mencionada, sino también cómo es que influyen las variables que
emplearemos como regresores, en la predicción final. Por ejemplo, si su
relación es directa o inversamente proporcional.

A continuación justificamos el posible uso de los caracteres presentes
en los datos, como covariables:

\begin{itemize}
\tightlist
\item
  \textbf{Sexo}: Debido al riesgo y costos distintos entre hombres y
  mujeres, para ciertos tipos de operaciones; por ejemplo, parto.
\item
  \textbf{Si fuma o no}: Pues fumar aumenta la probabilidad de
  desarrollar complicaciones médicas
\item
  \textbf{Región de la que provee}: Ya que el costo de un procedimiento
  médico puede variar mucho por región, así que también varía cuánto
  cubriría una aseguradora.
\item
  \textbf{Edad}: Puesto que pacientes mayores suelen requerir más
  cuidados.
\item
  \textbf{Índice de masa corporal}: En base a que un IMC elevado está
  asociado a mayores riesgos durante cirugías.
\item
  \textbf{Cantidad de hijos e hijas}: Esto puede influir en el tipo de
  cobertura familiar (de seguro) que tiene el paciente.
\item
  \textbf{Costo médico que pagaría en caso no se aplicase seguro
  médico}: Importante incluirlo, pues incluso se espera que presente una
  fuerte correlación positiva con la variable por predecir.
\item
  \textbf{Número de procedimientos pasados}: Puede resultar útil en base
  a que pacientes con muchos procedimientos suelen tener enfermedades
  crónicas, por lo que se esperaría una mayor cobertura.
\item
  \textbf{Número de pasos que realizó en cierto día}: Esta variable
  tampoco se explica en la fuente, pero la podemos considerar como una
  medida de la condición física de una persona, qué tan activa es.
\item
  \textbf{Número de veces que ha sido hospitalizado}: Pues más
  hospitalizaciones implican mayor riesgo en la operación, aumentando
  posiblemente así los costos que cubre la aseguradora.
\item
  \textbf{Salario anual}: Como indicador de nivel socioeconómico, se
  espera que pacientes con ingresos altos cuenten con aseguradoras que
  cubren mayor parte el costo por intervención médica.
\end{itemize}

\subsection{Itinerario metodológico de la
modelización}\label{itinerario-metodoluxf3gico-de-la-modelizaciuxf3n}

A continuación, describrimos los pasos a seguir para la construcción de
diferentes modelos de predicción:

\begin{enumerate}
\def\labelenumi{\arabic{enumi}.}
\tightlist
\item
  hola
\item
  hola
\end{enumerate}

\section{Materiales y métodos}\label{materiales-y-muxe9todos}

\subsection{Descripción genérica de los
datos}\label{descripciuxf3n-genuxe9rica-de-los-datos}

\section{Estructura de los datos}\label{estructura-de-los-datos}

A continuación, mostramos los datos descargados del sitio web mencionado
en la sección previa.

\subsection{Variables iniciales}\label{variables-iniciales}

\begin{itemize}
\tightlist
\item
  Variables \textbf{cualitativas}

  \begin{itemize}
  \tightlist
  \item
    \texttt{sex}: Sexo.
  \item
    \texttt{smoker}: Si el paciente fuma o no.
  \item
    \texttt{region}: Región de la que provee el paciente.
  \end{itemize}
\item
  Variables \textbf{cuantitativas}

  \begin{itemize}
  \tightlist
  \item
    \texttt{age}: Edad.
  \item
    \texttt{bmi}: Índice de masa corporal.
  \item
    \texttt{children}: Cantidad de hijos e hijas.
  \item
    \texttt{Claim\_Amount}: Costo médico que el paciente pagaría en caso
    no se aplicase seguro médico.
  \item
    \texttt{past\_consultations}: Número de procedimientos pasados del
    paciente.
  \item
    \texttt{num\_of\_steps}: Número de pasos que realizó el paciente en
    cierto día.
  \item
    \texttt{Hospital\_expenditure}: Gastos del hospital debido al
    procedimiento médico.
  \item
    \texttt{Number\_of\_past\_hospitalizations}: Número de veces que el
    paciente ha sido hospitalizado.
  \item
    \texttt{Anual\_Salary}: Salario anual.
  \item
    \texttt{charges}: \textbf{Pago final} que el paciente realizó por el
    procedimiento médico, tras haberse descontado el monto que cubre el
    seguro médico.
  \end{itemize}
\end{itemize}

\textbf{Carácter respuesta}: \texttt{charges}

Debido a la limitación, para esta investigación, de máximo 10 variables,
además de solo una o dos variables cualitativas, ignoraremos algunas
variables para este trabajo.

\subsection{Filtro de variables}\label{filtro-de-variables}

Para el filtro de variables categóricas, descartaremos aquella para la
cual las distribuciones de la variable respuesta, respecto a los valores
de aquella variable categórica sean relativamente similares.

\subsubsection{\texorpdfstring{Variable cualitativa
\texttt{region}}{Variable cualitativa region}}\label{variable-cualitativa-region}

Inspeccionamos la distribución de la variable respuesta, respecto a los
valores de la variable categórica \texttt{region}.

En base a que aquellas funciones densidad no presentan una difencia
resaltante, descartaremos la variable \texttt{region}. De esa manera,
las variables cualitativas que emplearemos para esta investigación son
solo \texttt{sex} y \texttt{smoker}.

Por otro lado, en el caso de las variables cuantitativas, primero
inspeccionamos la correlación entre aquellas. Esto para descartar alguna
de las variables que presente (de ser el caso) alta correlación lineal
con otra variable cuantitativa.

A continuación, presentamos un heatmap de aquella matriz de
correlaciones:

Observamos una alta correlación entre \texttt{Anual\_Salary} y
\texttt{Hospital\_expenditure}, con un valor de 0.9692177. Asimismo,
como la variable de salario anual es más sencilla de recopilar (por
ejemplo, en una encuesta) que la de gasto de hospital, descartamos la
variable cuantitativa \texttt{Hospital\_expenditure}.

\subsubsection{\texorpdfstring{Variable cuantitativa
\texttt{num\_of\_steps}}{Variable cuantitativa num\_of\_steps}}\label{variable-cuantitativa-num_of_steps}

Inicialmente se consideró descartar la variable referente al número de
pasos que realizó el paciente en cierto día. Esto pues, a primera vista,
no se esperaría que tal información resulte relevante para el costo
final por el procedimiento médico.

Graficamos tal posible regreso contra la variable respuesta:

En base a que la relación parece asemejarse a una exponencial,
graficamos la variable \texttt{num\_of\_steps} contra el logaritmo de la
variable respuesta:

En base a que aquella relación parece ser \emph{aproximadamente} lineal,
optamos por no descartar la variable cuantitativa
\texttt{num\_of\_steps}.

\subsubsection{\texorpdfstring{Variable cuantitativa
\texttt{num\_of\_steps}}{Variable cuantitativa num\_of\_steps}}\label{variable-cuantitativa-num_of_steps-1}

\subsubsection{\texorpdfstring{Variable cuantitativa
\texttt{age}}{Variable cuantitativa age}}\label{variable-cuantitativa-age}

\subsubsection{\texorpdfstring{Variable cuantitativa
\texttt{bmi}}{Variable cuantitativa bmi}}\label{variable-cuantitativa-bmi}

\subsubsection{\texorpdfstring{Variable cuantitativa
\texttt{children}}{Variable cuantitativa children}}\label{variable-cuantitativa-children}

\subsubsection{\texorpdfstring{Variable cuantitativa
\texttt{past\_consultations}}{Variable cuantitativa past\_consultations}}\label{variable-cuantitativa-past_consultations}

Descartamos la variable cuantitativa \texttt{children}, pues, en base a
este simple análisis inicial, no parece indicar algún de tipo de
relación lineal con la variable por predecir. Es más, su gráfico de
dispersión parece sugerir que consideremos a la variable
\texttt{children} como cualitativa.

\subsection{Variables finales}\label{variables-finales}

\begin{itemize}
\tightlist
\item
  Variables cualitativas:

  \begin{itemize}
  \tightlist
  \item
    \texttt{sex}
  \item
    \texttt{smoker}
  \end{itemize}
\item
  Variables cuantitativas:

  \begin{itemize}
  \tightlist
  \item
    \texttt{age}
  \item
    \texttt{bmi}
  \item
    \texttt{Claim\_Amount}
  \item
    \texttt{past\_consultations}
  \item
    \texttt{num\_of\_steps}
  \item
    \texttt{Number\_of\_past\_hospitalizations}
  \item
    \texttt{Anual\_Salary}
  \item
    \texttt{charges} (\textbf{variable respuesta})
  \end{itemize}
\end{itemize}

\section{Base de datos}\label{base-de-datos}

La base de datos consiste de 1338 observaciones. Si eliminamos filas que
posean algún dato vacío, se tienen 1287 observaciones.

Para limitarnos a 500 filas, realizaremos un muestreo:




\end{document}

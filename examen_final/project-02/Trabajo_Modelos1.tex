\documentclass[12pt]{article}
\usepackage[utf8]{inputenc}
\usepackage[spanish]{babel}
\usepackage{graphicx}
\usepackage{amsmath}
\usepackage{hyperref}
\hypersetup{
    colorlinks=true,
    linkcolor=blue,
    filecolor=magenta,      
    urlcolor=cyan,
    pdftitle={Trabajo de Aplicación},
    pdfpagemode=FullScreen,
    }
\usepackage{xcolor}
\usepackage[a4paper, total={6.4in, 9in}]{geometry}
\usepackage{setspace}
\spacing{1.5}

\begin{document}

\begin{titlepage}
\centering
{\bfseries\large Pontificia Universidad Católica del Perú \par}
\vspace{1cm}
{\scshape\large Maestría en Estadística \par}
\vspace{1cm}
{\scshape\large Trabajo de Aplicación: \textcolor{red}{aquí un título apropiado} \par}

\rule{150mm}{0.1mm}\\  
\vspace{3cm}
{\large Curso: Modelos lineales 1 \par}
\vfill
{\large Autor: \par}
{\large Nombre Apellidos \par}
\vfill
{\large Profesores: \par}
{\large Camiz Sergio \par}
{\large Alex de la Cruz Huayanay \par}
\vfill
{\large Junio 2025 \par}
\end{titlepage}

% Resumen
\section*{Indicaciones generales}

\noindent El informe debe tener una extensión máxima de 12 páginas, distribuidas de la siguiente forma:  
    \begin{itemize}
        \item Máximo 1 página de introducción,
        \item Máximo 2 páginas y media de metodología,
        \item Máximo 6 páginas de presentación y análisis de resultados,
        \item Máximo 1 página de comentarios o conclusiones finales.
        \item Referencias.
    \end{itemize}

\begin{itemize}
    \item A lo largo del informe no debe colocar salidas de R (ni capturas de pantalla o códigos).
    
    \item El trabajo final debe entregarse en formato PDF y se debe incluir el archivo en LaTeX o word.

    \item Se debe adjuntar la base de datos utilizada para el análisis.

    \item Se debe adjuntar todo el código en \texttt{R} empleado para la elaboración del informe.
\end{itemize}

\newpage

% Índice
\tableofcontents

\newpage

% Introducción
\section{Introducción}
En esta sección, el estudiante debe:  
\begin{itemize}
    \item Justificar la elección de la base de datos y describir explicando su contexto. 
    \item Especificar la variable respuesta y las variables explicativas.  
    \item Plantear los objetivos del análisis indicando de forma clara qué relación o patrones espera investigar.
\end{itemize}

\noindent \textbf{Nota:} Seleccionar una base de datos pública de una fuente conocida (UCI, Kaggle, INEI u otra).  


% Metodología
\section{Metodología}
En esta sección se espera que el estudiante detalle los pasos cómo implementará cada parte del análisis. Aquí solo va la metodología y no los resultados. Debe incluir:
\begin{itemize}
    \item Análisis Exploratorio de Datos (EDA): qué usará y qué espera analizar.
    \item Selección de Variables: qué tipos de selección usará y qué criterio (AIC, BIC, $R^2$ ajustado) (solo uno o dos).  
    \item Regularización: Presentar los modelos Ridge y Lasso, para el contexto de sus datos. Qué librerías usará y cómo comparará resultados con la selección clásica.  
    \item Modelos de Regresión: Presentar modelo AR(1), modelo con heterocedasticidad y modelo de regresión robusta en el contexto de sus datos. Explicar cómo verificará la auto-correlación para justificar el uso o no de un modelo AR(1) o un modelo ponderado para corregir heterocedasticidad. Qué paquetes usará.
\end{itemize}

% Resultados
\section{Resultados}
En esta sección se espera que el estudiante presente:
\begin{itemize}
    \item Descripción de los Datos: Tablas y gráficos principales del EDA.  
    \item Selección de Variables: Resultados de la selección de subconjuntos y Stepwise, tablas de modelos comparativos, variables retenidas y justificación.  
    \item Regularización: Coeficientes estimados para Ridge y Lasso para distintos valores de penalización, comparando con los modelos clásicos.
    \item Modelos AR(1) y Ponderado: Ecuación final de los modelo ajustados, parámetros estimados, diagnóstico de supuestos y comparación con el modelo usando regularización y MCO. Elegir un modelo entre todos los propuestos.
    \item Modelos de regresión robusta: para el mismo conjunto de datos, analizar modelo de regresión robusta y comparar con el modelo elegido en el ítem anterior. Eligir el mejor modelo.
    \item Validación del modelo final: Análisis de residuos, gráficos de ajuste y métricas de calidad del modelo.
\end{itemize}

% Discusión
\section{Comentarios finales}
En esta sección se espera que el estudiante:
\begin{itemize}
    \item Destaque la utilidad del mejor modelo en contraste con los otros modelos aplicados.
    \item Identifique limitaciones del análisis y del conjunto de datos. 
\end{itemize}



\section*{Referencias}

Las más principales

\end{document}
